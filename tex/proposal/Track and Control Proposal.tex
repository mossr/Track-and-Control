\documentclass[12pt]{article}
\usepackage{amsmath}
\title{Track and Control}
\author{Robert Moss, Aaron Pereira, Matthew Shrago}
\date{\today}

\usepackage{listings}
\usepackage{wrapfig}
\usepackage{graphicx}

\addtolength{\oddsidemargin}{-.875in}
\addtolength{\evensidemargin}{-.875in}
\addtolength{\textwidth}{1.75in}
\addtolength{\topmargin}{-.875in}
\addtolength{\textheight}{1.75in}


\begin{document}

\maketitle

\section{Proposal}

This software will allow users to organize, manipulate, and manage their OS windows and other features by moving their head. The tracking module will take care of tracking and calculating the position of their head via OpenCV. While the windows manager module will supply the user with an assortment of features to help organize their operating system's real estate. A local back-end database will store the user's settings for each feature.

Features include, a proportional snap to grid set up for users to place desired windows in five different locations. Namely, left, right, top, bottom, and middle. Once all the windows are selected, the middle window will fill the screen and the user will be able to "peer" in the direction of the four other windows to show a preview of them. This helps when the user is on a computer with a small screen, to truly maximize their space. Another feature would detect when the user is away from the screen, and with a user defined delay, put the computer to sleep, or simply turn off the display. Similarly to the window grid, the software can organize all open windows into a 3D view (with the illusion of layered windows) for the user to "look" around their desktop and see which window they want to select. All of these features will be activated with a user defined hot key. The modularity of the software allows us to easily update our tracking algorithm and include additional features.


\end{document}